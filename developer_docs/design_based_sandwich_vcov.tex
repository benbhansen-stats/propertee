We've promised to provide sandwich-type SEs for estimates of treatment
effect, with stacking of estimated equations as appropriate to address
possible use of separate regression fits to estimate covariance
adjustment terms and then to estimate marginal treatment effects.
Often ``sandwich standard error'' conjures an expectation of standard
errors that are model-based in the sense of assuming i.i.d.  data
\((A_i,  W_{i}, X_i, Y_{0i}, \ldots Y_{Ki})\) and making inference conditional
on the values of treatment assignments \(A \in
\{0, \ldots, K\}\), importance or frequency weights \(W\), and covariates \(X\), but
not the values of potential responses \(\{Y_k: k \leq K\}\).  (This
document takes $i$ to range over intact clusters.)  With the i.i.d assumption, these model-based SEs are valid
even if the design specification isn't, provided that the outcome
model is well-specified. The role of the study design is to provide
analysis weights which, if respected, ensure desired interpretations
of marginal effect estimators if the \(Y\)-model happens to be
misspecified.

An alternative would be to offer sandwich standard errors under a design-based
interpretation, differing from the above principally in terms of
conditioning on

\[\mathcal{Z} = \sigma(\{(X_i, Y_{0i}, \ldots Y_{Ki}): i\}, \{\sum_{i
    \in S}\, \![A_i=k\!] : k \in \{0, \ldots, K\}; S  \in \mathcal{P}\} )\]

where \(\mathcal{P}\) is a given partition of \{1,\ldots{}, n\} encoding
a stratification. Of particular interest are standard errors for Hajek
estimators, i.e. estimators of form
\[\hat{\mu}_k[\mathbf{v}] = \frac{\sum_i \check{w}_{i[k]} [\![A_i=k]\!] v_i}{\sum_i \check{w}_{i[k]}  [\![A_i=k]\!]}\]
and their differences
\begin{equation}\label{eq:hajekdiff}
\hat{\mu}_k[\mathbf{u}] - \hat{\mu}_0[\mathbf{v}].
\end{equation}
The case weights $\{\check{w}_{i[k]}:i\leq n, k\leq K\}$ are the products of
a frequency or importance weight $w_{i}$ and, as
appopriate, a weighting factor representing a reciprocal of $i$'s
probability (or odds) of assignment to $k$ (or to one of \{$k, \ldots,
K$\}); in general differ by $k$.
With $\check{w}_{i[j]} =[\mathbb{P}(A_{i}=j \mid \mathcal{Z}]^{-1}$ for
$j=0, \ldots, K$,
$\hat{\mu}_k[\mathbf{y}_{k}] - \hat{\mu}_0[\mathbf{y}_{0}]$ is the
natural estimate of the effect of treatment $k$ within the
experimental or quasiexperimental study sample,
$n^{-1}\sum_{i=1}^{n}(y_{ki}-y_{k0})$.  This is also known as the
sample average treatment effect. If the experimental observations are
samples from a broader super-population, with sample inclusion
probabilities $\pi_{1}, \ldots, \pi_{n}$, then setting
$\check{w}_{i[j]} =\pi_{i}^{-1}[\mathbb{P}(A_{i}=j)]^{-1}$ for $j=0, \ldots, K$
makes $\hat{\mu}_k[\mathbf{y}_{k}] - \hat{\mu}_0[\mathbf{y}_{0}]$ into
the corresponding natural\footnote{Neither of these estimators is unbiased, except in special
cases where the conditioning and weighting structures are such that
$\mathrm{Var}(\sum_i \check{w}_{i[k]} [\![A_i=k]\!] \mid \mathcal{Z}) = \mathrm{Var}(\sum_i
\check{w}_{i[0]} [\![A_i=0]\!] \mid \mathcal{Z}) = 0$, but the biases are
in various senses controlled. Under the i.i.d. sampling model that
stands behind model-based variance estimates, for instance, they are
consistent; taking the design perspective, BH and MF have preliminary
findings about how to eliminate or control the biases through further
conditioning.} estimate of the population average treatment
effect.

\section*{Design-based SEs for differences of Hajek estimators}
\subsection*{A one-stratum special case: Neyman's insight}

In one special case a design-based SE for \eqref{eq:hajekdiff} has been available for some time, with an optional interpretation in terms of M-estimation emerging more recently.  Suppose:
\begin{itemize}
\item  simple or complete random assignment within a single
stratum, \(\mathcal{P} = \{\{1,\ldots,n\}\}\); with
\item constant weights, i.e. there exist $w_0$ and $w_k$ s.t. $\check{w}_{i[k]}= w_k$ and $\check{w}_{i[0]}= w_0$ for each $i$; and
\item  \(\sum_i [\![a_i = k]\!]\geq 2\) and \(\sum_i [\![a_i = 0]\!] \geq 2\).
\end{itemize}
In this case the variance estimate used
in the 2 sample \(t\) test without pooling of variances validly
estimates the variance of \eqref{eq:hajekdiff}, in the following sense. If \(v\) is observed
only when \(A=k\) while \(u\) is observed only when \(A=0\), then
\(s_{vu} = (n-1)^{-1} \sum_{i=1}^n (v_i - \bar v)(u_i - \bar v)\) is
not identified; as far as the data are concerned, it could lie anywhere
between \(\pm s_{v} s_{u}\), where
\(s_{v}^2 = s_{vv} = (n-1)^{-1} \sum_{i=1}^n (v_i - \bar v)^2\). Assuming
\(s_{vu} = s_{v}s_{u}\), then the ordinary unpooled variance is unbiased for
the design-based variance (conditional variance given \(\mathcal{Z}\))
of \eqref{eq:hajekdiff}. If on the other hand \(s_{uv} \leq s_{v}s_{u}\), then the expected value of this ordinary
unpooled variance exceeds \eqref{eq:hajekdiff}'s design-based variance
\citep[][A32--34]{neyman:1990,freedman:purv:pisa:1998}. The
possibilities \(s_{vu} = s_{v}s_{u}\) and \(s_{vu} \leq s_{v}s_{u}\) being
exhaustive, by Cauchy-Schwartz, the expected value of the unpooled
variance can be no less than $\mathrm{Var}(\hat{\mu}_k[\mathbf{u}] -
\hat{\mu}_0[\mathbf{v}] \mid \mathcal{Z})$; it is valid as a possibly
conservative estimate in that it cannot be negatively biased.

As shown by \citet{samiiAronow2012HC2equivNeyman}, in
this case the same variance estimate coincides with the HC2 flavor of
the Huber-White estimate.  As a result, it's sometimes said that
ordinary sandwich or cluster-robust variance estimates admit of
design-based interpretations provided that one uses the HC2
form. I suspect this works only in special cases, however, perhaps not going far
beyond the one delimited above.

\subsection*{Impediments to application of the Neyman insight to
  experiment designs beyond the 1-stratum special case}

The use of the Neyman insight to get design-based SEs through
Huber-White covariance estimates intended for model-based interpretation
has several important limitations.

\begin{enumerate}
\def\labelenumi{\arabic{enumi}.}
\item \label{it:prob-noweights}
  The unequal variances two-sample \(t\) statistic doesn't in itself accommodate
  clustering.  In particular, it doesn't allow weights figuring in \eqref{eq:hajekdiff},
$\{\check{w}_{i[0]} :i\}$ or  $\{\check{w}_{i[k]} :i\}$, to vary, as is necessary
to address clustered treatment assignment by letting $v_{i}$, $u_{i}$,
$y_{ki}$ and so on
represent means over all observations clustered within unit of
assignment $i$.
\item \label{it:moderators-multiple-outcomes}  It remains to be
  articulated how the Neyman insight can be made to furnish design-based SEs
  for effect estimates specific to a subgroup or domain. (E.g., ``What was
  the benefit for girls?'') How such
  effect estimates covary with main effect
  estimates and with one another also remains to be articulated, as
  does the covariance between main effects on distinct outcomes. Both
  of these are important for multiplicity adjustments.
\item \label{it:prob-multreps}
  The unequal variances two-sample \(t\) statistic requires that the two
  conditions being contrasted each have multiple representatives within
  each stratum. When this doesn't hold, as in matched designs, it's
  unclear what it can be replaced with.
\item \label{it:prob-covars}
  Samii and Aronow's result does not allow for covariates other than the
  treatment assignment variable. In a two-group setting with covariates
  as well as a treatment indicator, does HC2 continue to be valid as a
  design-based SE? Under what conditions? I suspect the answer is either
  no, or yes but only under restrictive conditions.\footnote{Samii and
    Aronow didn't discuss this and I couldn't immediately find it in
    later papers citing theirs. There is a Abadie et al 2021 paper that
    may indirectly address this, but I suspect that if they'd found a
    standard model-based Huber-White estimator to have a valid
    design-based interpretation, they would have said just that.}
  If this is correct,  there's still a need for a general-purpose SE
  estimator for effect estimates calculated with
covariate adjustments; HC2 does not fill this role.
\item \label{it:prob-differentstatuses}
  In the general setting of chained estimation ---  one or more
  preliminary regressions are fit, to the experimental sample, to a
  separate and disjoint sample or to some combination of the two,
  followed by a difference-of-Hajek estimates step involving residuals
  of these preliminary regressions' residuals --- off-the-shelf
  sandwich-type variance estimates such as HC2 can't admit a design-based interpretation as well as
  a  model-based one. The calculation would have to recognize differing statuses of observations within and outside of the experimental or quasiexperimental sample, something the conventional estimates don't do.
\item \label{it:prob-nonexperimentals-as-nonrandom}
  When the covariance adjustment model has been fit to an extraneous or
  broader sample, the natural extension of \(\mathcal{Z}\) regards
  observations external to the experimental sample as nonrandom, meaning
  that they shouldn't contribute to sampling variability.\footnote{Just as in
  RDD analysis, data points $(X_{i}, A_{i}, Y_{0i}, \ldots, Y_{Ki})$
  falling outside of the analysis window are
  conditioned upon and then regarded as nonrandom \citep[][display A1 on p.22]{salesHansen2019lrd}.} This should decrease standard
  errors relative to model-based SEs, to which the extraneous observations should make a
  contribution.
\item \label{it:quantile-regr-SEs}  We proposed to explore the use of
  quantile regression in RDDs, as an alterative to robust estimation
  of the forcing variable's slope via bounded influence robust
  regression. (Quantile regression lacks the bounded influence
  property in general, but with forcing variables limited to a bounded
  range it may not.) Under the ordinary model-based interpretation
  with $Y$ random while covariate and treatment variables are
  considered fixed, Huber-White estimation is not generally used,
  because it calls for estimation of the reciprocal of a conditional density of $Y
  \mid X$, the ``sparsity function,'' and depends sensitively on the estimated
  value \citep[][\S3.4]{koenker2005quantileregr}. Estimates of a quantile regression's B matrix would be
  difficult to find in off-the-shelf software \citep[][\S2.5]{he18resampling}.
\end{enumerate}

\section*{The promise of M-estimation}
The M-estimation framework promises solutions to many or all of
these impediments.
\paragraph*{An immediate solution to
  Impediment~\ref{it:prob-noweights} and a part of Impediment~\ref{it:moderators-multiple-outcomes}}

The Hajek estimators $\hat{\mu}_{0}[\mathbf{y}_{0}]$, \ldots,
$\hat{\mu}_{k}[\mathbf{y}_{k}]$ admit of being defined as the unique
multivariate root of the estimating function
\begin{equation} \label{eq:ee-hajekdiff}
(\mu_{0}, \ldots, \mu_{K}) \mapsto
  \left[
\begin{array}{c}
\sum_i [\![a_i=K]\!] \check{w}_{i[K]} (y_{Ki} - \mu_K) \\
\vdots \\
\sum_i [\![a_i=0]\!] \check{w}_{i[0]} (y_{0i} - \mu_0) 
\end{array}
\right]
.
\end{equation}

The information matrices $A(\mu_0, \ldots, \mu_K)$ corresponding to
\eqref{eq:ee-hajekdiff} are diagonal.  For a sandwich estimate of
variance, then, we only need estimates of the covariance
of~\eqref{eq:ee-hajekdiff}.  In the one-stratum case with multiple
representatives of conditions $k$ and $0$, we can simply apply
Neyman's two-sample unpooled variances SE to observations $\{ \check{w}_{i[k]}
(y_{ki} - \mu_k) : A_i = k\}$ and $\{ \check{w}_{i[0]} (y_{0i} - \mu_0) : A_i =
0\}$.  Impediment~\ref{it:prob-noweights} is solved.

For standard errors of the difference of Hajek estimators within a
subgroup $g$,
\begin{equation*}
 \frac{\sum_i \check{w}_{i[k]} [\![A_i=k, G_{i} =g]\!] y_{ki}}{\sum_i \check{w}_{i[k]}
     [\![A_i=k, G_{i}=g]\!]}
 -
  \frac{\sum_i \check{w}_{i[0]} [\![A_i=0, G_{i} =g]\!] y_{0i}}{\sum_i \check{w}_{i[0]}
     [\![A_i=0, G_{i}=g]\!]},
\end{equation*}
we can simply fold the subgroup indicator into the weights,
weighting by $\{\tilde{w}_{i[j]}:j; i\} = \{\check{w}_{i[j]}[\![G_{i}=g]\!]: j; i\}$ as opposed to $\{\check{w}_{i[j]}:j;
i\}$. This solves the first part of
Impediment~\ref{it:moderators-multiple-outcomes}.

(The remaining piece of Impediment~\ref{it:moderators-multiple-outcomes}
is the need for estimates of
\begin{equation*}
  \mathrm{Cov}\left(
    \frac{\sum_i \check{w}_{i[k]} [\![A_i=k]\!] v_{ki}}{\sum_i \check{w}_{i[k]}
      [\![A_i=k]\!]} -
    \frac{\sum_i \check{w}_{i[k]} [\![A_i=k]\!] v_{0i}}{\sum_i \check{w}_{i[k]}  [\![A_i=k]\!]},
  \frac{\sum_i \check{w}_{ki} [\![A_i=k]\!] u_{ki}}{\sum_i \check{w}_{ki}
      [\![A_i=k]\!]} -
    \frac{\sum_i \check{w}_{ki} [\![A_i=k]\!] u_{0i}}{\sum_i \check{w}_{ki}  [\![A_i=k]\!]}
  \right)
\end{equation*}
where $v_{ji}$ and $u_{ji}$ are observed only
when $A_{i}=j$, all $i$ and $j$.   I haven't had the opportunity to
sit down and try to work this out, but I'm optimistic that this
can be handled with straightforward extensions of the Neyman insight.)

\paragraph*{A near-immediate solution to
  Impediment~\ref{it:quantile-regr-SEs}}
Under the design-based interpretation, variances and covariances of
quantile regression estimating functions --- for a quantile regression
with covariates $X$ and $c_{i}$ observations per cluster $i$,
\begin{equation*}
\beta \mapsto  \sum_{i=1}^{n}\sum_{j=1}^{c_{i}}\sum_{k=0}^{K} x_{ij}([\![y_{kij} -
  X_{ij}'\beta, a_{i}= k)]\!] - \tau)
\end{equation*}
\citep[][\S2.7]{he18resampling} --- are likely to be estimable in
much the same way as variances and covariances of estimating functions
for other forms of regression.  In any case, there will be no need for
estimation of a sparsity parameter.  The impediment is removed.

(Quantile regression doesn't bear any of its usual interpretations
under conditioning on $\mathcal{Z}$, but that doesn't matter for
purposes of using it to estimate slopes in an RDD.)

\paragraph*{Sketch of a solution to Impediment~\ref{it:prob-multreps}}
Now consider Impediment~\ref{it:prob-multreps},  strata $S$ in which one
or both of the conditions $j$ being contrasted have only one
representative, $\sum_{i\in S} [\![A_{i}=j]\!] =1$. Consider the situation without covariates.

As there is independence across if not within strata, the
B matrices corresponding to~\eqref{eq:ee-hajekdiff} are sums of
stratum-wise contributions
\begin{equation} \label{eq:Bform}
B_S((\mu_0, \ldots, \mu_K)) =
\mathrm{Cov}\left(\left[
\begin{array}{c}
\sum_{i\in S} [\![A_i=K]\!] \check{w}_{i[K]} (y_{Ki} - \mu_K) \\
\vdots \\
\sum_{i\in S} [\![A_i=0]\!] \check{w}_{i[0]} (y_{0i} - \mu_0)
\end{array}
\right] \mid \mathcal{Z}\right), \quad S \in \mathcal{P}.
\end{equation}
For any stratum $S$ and condition $k$ s.t.  $\sum_{i \in S}  [\![A_i=k]\!] =1$, $\mathrm{Var}\left\{ \sum_{i\in S} [\![A_i=k]\!] \check{w}_{i[k]} (y_{ki} - \mu_k) \right\}$ is not estimable, just as the variance $[(\# S)-1]^{-1}\sum_{i \in S} (y_{ki} - \bar{y}_k)^2$ is not estimable. On the other hand, the second moment
\[ \mathbb{E}\left[\left\{ \sum_{i\in S} [\![A_i=k]\!] \check{w}_{i[k]} (y_{ki} - \mu_k) \right\}^2\right]
\]
\textit{is} estimable, by it sample realization \(\left\{ \sum_{i\in
    S} [\![a_i=k]\!] \check{w}_{i[k]} (y_{ki} - \mu_k) \right\}^2\): in
M-estimation, we're entitled to treat $\mu_k$ is a fixed constant, not
a random variable. Since the second moment  bounds the variance from above,  \(\{\sum_{i\in S} [\![A_i=k]\!] \check{w}_{i[k]} (y_{ki} - \mu_k)\}^2\) is a safe, conservative variance estimate.

M-estimation doesn't help us to estimate
\[
\mathrm{Cov}\left\{
\sum_{i\in S} [\![A_i=k]\!] \check{w}_{i[k]} (y_{ki} - \mu_k),
\sum_{i\in S} [\![A_i=0]\!] \check{w}_{i[0]} (y_{0i} - \mu_0)
\right\}
\]
but these terms aren't identified even for large strata with multiple
representatives of either condition. When interest is in the
difference $\hat{\mu}_k - \hat{\mu}_0$, conservatism leads us to
``impute'' the covariance value that maximizes the variance of that difference, that
corresponding to a correlation of 1 between $\sum_{i\in S} [\![A_i=k]\!] \check{w}_{i[k]}
(y_{ki} - \mu_k)$ and $\sum_{i\in S} [\![A_i=0]\!] \check{w}_{i[0]} (y_{0i} -
\mu_0)$. We can follow the same principle when approaching the problem
from the perspective of $M$-estimation, whatever the representation of
conditions $k$ and 0 within stratum $S$.

In separate hand-written
notes I begin fleshing this out into methods of estimating design-based second
moments of first differences of the estimating functions
\eqref{eq:ee-hajekdiff} that define the Hajek estimators, which
can in turn be used to approximate the corresponding B-matrix terms;
extending this work might be a suitable problem for a PhD student.

\begin{quote}
\textit{Code design notes:}\\
1. Because we don't really estimate
\begin{equation*}
  \mathrm{Cov}\left\{
\sum_{i\in S} [\![A_i=k]\!] \check{w}_{i[k]} (y_{ki} - \mu_k),
\sum_{i\in S} [\![A_i=0]\!] \check{w}_{i[0]} (y_{0i} - \mu_0)
\right\},
\end{equation*}
but only impute this covariance based on
$\sum_{i\in S} [\![A_i=k]\!] \check{w}_{i[k]} (y_{ki} - \mu_k)$'s and
$\sum_{i\in S} [\![A_i=0]\!] \check{w}_{i[0]} (y_{0i} - \mu_0)$'s estimated
variances and presumed correlation of 1, we should consider software design that doesn't tempt the user into
treating our stand-in for it as a real estimate, perhaps in some
unanticipated situation.  This could be accomplished through a
function returning differences of Hajek
estimators~(\ref{eq:hajekdiff}), i.e. estimates of $\mu[\mathbf{y}_{k}] -
\mu[\mathbf{y}_{0}]$ for each $k$, but not the separate Hajek estimators
$\hat{\mu}_{k}[\mathbf{y}_{k}]$ or $\hat{\mu}_{0}[\mathbf{y}_{0}]$, nor an
intercept. The user then expects variances and covariances of these
differences of Hajek estimators, which we can legitimately provide,
but not the covariances of the Hajek estimators themselves, which we
formally compute but aren't really what they would appear to be.  This
would in turn call for using our own direct adjustment/Hajek
difference function to furnish estimates, not \texttt{lm()}, which
would also report material we don't wish to be responsible for
providing standard errors for.  (If $A$ is encoded as a factor with 0
as its reference level, then perhaps this is a wrapper to
\texttt{lm()} that strips out the intercept and intercept-related rows
and columns of the information matrix prior to returning.)\\
2. The upshot of this will be a novel method of B-matrix estimation,
applicable to estimating equations arising from both the ultimate
Hajek estimation and earlier-stage covariance estimation, i.e. early as well as final
steps of the estimator chain.\\
3. Current literature \citet[e.g.,][\ldots]{mukerjeeDasguptaRubin18,pashleymiratrix18varestimationblockeddesigns} may well present other design-based covariance
estimators worth implementing as alternative B-matrix estimators.
It'd be best to design in such a way as to facilitate alternating
among B-matrix estimation routines.
\end{quote}
%<!--For a stratum $S$ s.t. $\sum_{i \in S}  [\![A_i=k]\!] =  \sum_{i
%\in S} [\![A_i=0]\!]=1$, e.g. a matched pair in which conditions 0 and $k$ are represented.--> 

\paragraph{Addressing Impediment~\ref{it:prob-covars}}
If covariate \(X\)  are included in a linear regression specification
\verb|y ~ a + x| with
unit weights $\sum_{j=1}^{K}\check{w}_{i[j]}[\![a_{i}=j]\!]$ --- that is, the
specification that without \(X\) would have engendered estimation
of Hajek estimator differences as in \eqref{eq:ee-hajekdiff} ---
then B-matrix estimation can still be done just
as it is without the covariates.  The A matrices are no longer
diagonal, but are otherwise straightforward.  Combining estimates of
these A and B matrices would give legitimate standard errors for
differences of Hajek estimators as before, but now with adjustment for
covariates\footnote{Formally it would also seem to furnish SEs for the
covariate effects themselves; I haven't thought through whether these
would indeed be interpretable as design-based SEs.%
}.

%<!-- init segment of document shared w/ Xinhe W, early Feb22, ends here-->

Another approach to covariates is to have fit their coefficients in a separate and
prior regression fit --- perhaps a robust linear fit; or perhaps a binary
regression fit --- and then to contrast Hajek aggregates of residuals
from these fits \citep{gelmanPardoe07,bowers:hans:2008}.  This approach calls for
stacked estimating equations, with accompanying complications to both
the A and the B matrix. We might as well consider it in tandem with
challenges \ref{it:prob-differentstatuses} and
\ref{it:prob-nonexperimentals-as-nonrandom}.

\paragraph{Toward solutions to Impediments (\ref{it:prob-differentstatuses}) and
  (\ref{it:prob-nonexperimentals-as-nonrandom})}


To draw a distinction between experimental versus non-experimental
observations, let $\mathcal{P}$ be a partition of a nonempty subset of
$\{1, \ldots, n\}$, so that some but not necessarily all clusters
1,\ldots, $n$ fall into some stratum $S \in
\mathcal{P}$; these clusters are then the experimental or
quasi-experimental sample. Re-define $\mathcal{Z}$ to contain similar information
as above but also full information on clusters falling outside of
$\mathcal{E} = \bigcap_{S \in \mathcal{P}}S$:
\begin{equation} \label{eq:Zdefextended}
  \mathcal{Z} = \sigma\left(\begin{array}{c}
                              \{(X_i, Y_{0i}, \ldots Y_{Ki}): i \in
                              \mathcal{E}\},
                              \{\sum_{i \in S}\, \![A_i=k\!] :
                              k \in \{0, \ldots, K\}; S  \in \mathcal{P}\} , \\
                              \{(A_{i}, X_i, Y_{0i}, \ldots
                              Y_{Ki}): i \not\in\mathcal{E}\}
                              \end{array}\right).
\end{equation}
One now needs A and B matrices for estimating functions of the Hajek
estimator and the regression estimator simultaneously.  Given that
we're ultimately interested only to estimate covariances for the Hajek
differences, simplified expressions in terms of A- and B-submatrices
are available \citep[e.g.][p.373 of \S~A.6.6.]{carroll2006measurement}.  Further
simplification of B matrices is possible given our conditioning on
$\mathcal{Z}$ in \eqref{eq:Zdefextended}; the B matrix continues to be
a sum of form~\eqref{eq:Bform}, with clusters $i\not\in \mathcal{E}$
making no contribution. (The design-based interpretation has no bearing on
calculation of \(A\) matrices.)

\begin{quote}
  \textit{Code design note:}
  When the final comparison of Hajek estimates builds upon a chain of
  earlier estimates, design-based standard errors require information
  from the earlier-stage fit just as model-based covariance estimates
  would, but there's an important different in the information that's
  required. From either perspective  information or A matrices from
  earlier steps of the chain have to be passed forward, or at least
  certain submatrices of inverted information matrices.  But the
  model-based perspective calls for much more about the B matrices.
  Fro example, if earlier estimators in the chain were fit to an
  external sample, then conditional on $\mathcal{Z}$ these estimates
  are nonrandom, and the corresponding $B$ matrices are 0, and can be
  ignored.  If the sample used for the earlier stage fit overlaps with
  the experimental sample, then from the design-based perspective it
  does have a B-matrix, but only the experimental observations
  contribute to it.

  Accordingly, rather than passing forward B matrices from earlier
  stage fits, we might instead pass forward:
  \begin{itemize}
  \item Sufficient information about the observations used in the
    fitting to precisely identify regions of overlap between the
    sample used for the prior-stage fit and the experimental sample.
  \item Sufficient information about the prior-stage fitting routine
    to reconstruct specific observations' contributions to its estimating functions.
  \end{itemize}
  The latter is also necessary to address estimator chains model-based
  covariance estimation, in order to estimate $B_{12}$ submatrices
  \citep[p.373]{carroll2006measurement}.
\end{quote}

<!--
%%% Local Variables:
%%% mode: latex
%%% TeX-master: "design_based_SEs_"
%%% End:
-->